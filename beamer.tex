\documentclass{beamer}
%\usetheme{Berlin}
\usetheme{Boadilla}
%\usetheme{Dresden}
%\usetheme{AnnArbor}
%\usetheme{Bergen}
%\usetheme{Berkeley}
%\usetheme{CambridgeUS}
%\usetheme{Szeged}
%\usetheme{Singapore}
%\usetheme{Pittsburgh} title is placed on the right hand
%\usetheme{PaloAlto}





%\usepackage[no-math]{fontspec}

%\usecolortheme{beaver}
\setbeamercolor{itemize item}{fg=darkred!80!black}
\mode<presentation>
\usepackage{dcolumn}
\usepackage[utf8]{vietnam}
\title{My Presentation}

\subtitle{Using Beamer}
\author{Đồng Xuân Bá}
\institute{Bách Khoa Hà Nội}
\date{\today}
\begin{document}
 
     \begin{frame}{\textbf{Cohabitation in vietnames is quite pu}}
\end{frame}
 
\begin{frame}
\frametitle{Title}
Lorem ipsum dolor sit amet, consectetur adipisicing elit, sed do eiusmod tempor incididunt ut labore et dolore magna aliqua.
\end{frame}
\begin{frame}{we are the world}
    \begin{columns}
    \column{0.5\textwidth}
    ou may copy and \vspace{0.5cm}distribute verbatim copies of the Program’s source code as you receive it, in anymedium, provided that you conspicuously and appropriately publish on each copy an appropriatecopyright notice and disclaimer of warranty; keep intact all the notices that refer to this License andto the absence of any warranty; and give any other recipients of the Program a copy of this Licensealong with the Program.\vspace{0.25}ou may charge a fee for the physical act of transferring a copy, and you may at your option offerwarranty protection in exchange for a fee.2.You may modify your copy or copies of the Program or any portion of it, thus forming a work based onthe Program, and copy and distribute such modifications 
    \column{0.5\textwidth}
    ou may copy and distribute verbatim copies of the Program’s source code as you receive it, in anymedium, prosfsafaaaaaaaaaasdfa
    \end{columns}
\end{frame}
\begin{frame}
\frametitle{Outline}
\begin{itemize}
    \item hello
    \pause
    \time main idea
    \pause
    \begin{itemize}
        \item part 1
        \pause
        \item part 2
    \end{itemize}
\end{itemize}
\tableofcontents
\end{frame}
\begin{frame}{picture in my life}
    \begin{figure}
        \centering
        \includegraphics{Screenshot_2020-04-24 How to Write a Thesis in LaTeX pt 1 - Basic Structure.png}
        \caption{the king of animal }
        \label{fig:my_label}
    \end{figure}
\end{frame}
\begin{frame}[fragile]
\frametitle{An Algorithm For Finding Primes Numbers.}
\begin{verbatim}
int main (void){std::vector<bool> is_prime (100, true);
for (int i = 2; i < 100; i++)
if (is_prime[i]){std::cout << i << " ";
for (int j = i; j < 100; 
is_prime [j] = false, j+=i);
24
}return 0;}
\href{https://ctan.org/pkg/csvsimple?lang=en}{\texttt{csvsimple}} package:
\end{verbatim}
\begin{uncoverenv}<2>Note the use of 
\verb|std::|.
\end{uncoverenv}
\end{frame}
 \begin{frame}{hell}
 \frametitle{test}
 %\begin{columns}
 \begin{flushleft}
 
 
 
e Program a copy o
 
 
 \includegraphics[scale=0.5]{Screenshot_2020-04-24 How to Write a Thesis in LaTeX pt 1 - Basic Structure.png}
 \end{flushleft}
d% \end{columns}     
 \end{frame}
 
 \begin{frame}{aaskdlfk}
 \frametitle{sdfhsf}
 \begin{block}{Block Title}
Lorem ipsum dolor sit amet, consectetur adipisicing elit, 
sed do eiusmod tempor incididunt ut labore et 
dolore magna aliqua.
\end{block}
\begin{alertblock}{Block Title}
Lorem ipsum dolor sit amet, consectetur adipisicing elit, 
sed do eiusmod tempor incididunt ut labore et 
dolore magna aliqua.

\end{alertblock}
     \begin{definition}
A prime number is a number that...
\end{definition}

\begin{example}
Lorem ipsum dolor sit amet, consectetur adipisicing elit, 
sed do eiusmod tempor incididunt ut labore et
dolore magna aliqua.
\end{example}

 \end{frame}
 
 
 \begin{frame}{fsdfhas}
 \frametitle{fsdfsa}
 \begin{theorem}[Pythagoras] 
$ a^2 + b^2 = c^2$
\end{theorem}
\begin{corollary}
$ x + y = y + x  $
\end{corollary}
\begin{proof}
$\omega +\phi = \epsilon $
\end{proof}
     
 \end{frame}
 
 \begin{frame}[fragile]
\frametitle{Including Code}
\begin{semiverbatim}
\\begin\{frame\}
\\frametitle\{Outline\}
\\tableofcontents
\\end\{frame\}
\end{semiverbatim}
\end{frame}

 
     
 \begin{frame}
\frametitle{List}
\begin{itemize}
\pause
\item Point A
\pause
\item Point B
\begin{itemize}
\pause
\item part 1
\pause
\item part 2
\end{itemize}
\pause
\item Point C
\pause
\item Point D
\end{itemize}
\end{frame}


\begin{frame}
\frametitle{More Lists}
\begin{enumerate}[(I)]
\item<1-> Point A
\item<2-> Point B
\begin{itemize}
\item<3-> part 1
\item<4-> part 2
\end{itemize}
\item<5-> Point C
\item<6-> Point D
\end{enumerate}
\end{frame}

\begin{frame}
\frametitle{Overlays}
\only<1>{First Line of Text}

\only<2>{Second Line of Text}

\only<3>{Third Line of Text}
\end{frame}


\begin{frame}{dfsf}
    
\frametitle{
sdfsdafas}
\textbf<2>{Example Text}
\textit<2>{Example Text}
\textsl<2>{Example Text}
\textrm<2>{Example Text}
\textsf<2>{Example Text}
\textcolor<2>{orange}{Example Text}
\alert<2>{Example Text}
\structure<2>{Example Text}


\setbeamercovered{transparent}
\end{frame}
 
 \begin{frame}
\frametitle{Maths Blocks}
\begin{theorem}<1->[Pythagoras] 
$ a^2 + b^2 = c^2$
\end{theorem}
\begin{corollary}<3->
$ x + y = y + x  $
\end{corollary}
\begin{proof}<2->
$\omega +\phi = \epsilon $
\end{proof}
\end{frame}
 
\end{document}


